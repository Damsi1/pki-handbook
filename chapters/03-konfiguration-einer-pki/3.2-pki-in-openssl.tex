\section{Konfiguration einer PKI in OpenSSL}
\subsection*{Ordnerstruktur}

\begin{verbatim}
mkdir -p ~/pki-demo/{ca,server,crl}
cd ~/pki-demo
mkdir -p ~/pki-demo/{ca,server,crl}
cd ~/pki-demo
\end{verbatim}

Im Benutzerverzeichnis wurde folgender Projektordner angelegt.

\subsection*{Erstellung der Root-CA}

Zuerst wird eine Root-CA erstellt, die als Vertrauensanker dient.

\subsubsection*{Private Key}

\begin{verbatim}
openssl genrsa -out ca/ca.key 4096
openssl genrsa -out ca/ca.key 4096
\end{verbatim}

\subsubsection*{Self-signed Root-Zertifikat}

\begin{verbatim}
openssl req -x509 -new -nodes -key ca/ca.key -sha256 -days 3650 \
-out ca/ca.crt \
-subj "/C=AT/ST=Nieder sterreich/L=Mariazell/O=C/OU=C/CN=Christian Wunder Root CA"

openssl req -x509 -new -nodes -key ca/ca.key -sha256 -days 3650 \
-out ca/ca.crt \
-subj "/C=AT/ST=Nieder sterreich/L=Mariazell/O=C/OU=C/CN=Christian Wunder Root CA"
\end{verbatim}

Dieses Zertifikat wird später in die macOS-Keychain importiert, um von Chrome und Safari als vertrauenswürdig erkannt zu werden.

\subsection*{Erstellung des Server-Zertifikats}

\subsubsection*{Private Key}

\begin{verbatim}
openssl genrsa -out server/server.key 2048
openssl genrsa -out server/server.key 2048
\end{verbatim}

\subsubsection*{Konfigurationsdatei mit SAN}

\begin{verbatim}
cat > server/openssl_server.cnf <<'EOF'
[ req ]
default_bits        = 2048
prompt              = no
default_md          = sha256
distinguished_name  = dn
req_extensions      = v3_req

[ dn ]
C = AT
ST = Niederoesterreich
L = Mariazell
O = C
OU = C
CN = localhost

[ v3_req ]
subjectAltName = @alt_names

[ alt_names ]
DNS.1 = localhost
IP.1  = 127.0.0.1
EOF
\end{verbatim}

(Identischer Block wurde im Original erneut eingefügt.)

\subsubsection*{CSR (Certificate Signing Request)}

\begin{verbatim}
openssl req -new -key server/server.key -out server/server.csr \
-config server/openssl_server.cnf

openssl req -new -key server/server.key -out server/server.csr \
-config server/openssl_server.cnf
\end{verbatim}

\subsubsection*{Signieren durch die Root-CA}

\begin{verbatim}
openssl x509 -req -in server/server.csr -CA ca/ca.crt -CAkey ca/ca.key \
-CAcreateserial -out server/server.crt -days 825 -sha256 \
-extfile server/openssl_server.cnf -extensions v3_req

openssl x509 -req -in server/server.csr -CA ca/ca.crt -CAkey ca/ca.key \
-CAcreateserial -out server/server.crt -days 825 -sha256 \
-extfile server/openssl_server.cnf -extensions v3_req
\end{verbatim}

\subsection*{Import der Root-CA in macOS}

\begin{verbatim}
sudo security add-trusted-cert -d -r trustRoot \
-k /Library/Keychains/System.keychain ca/ca.crt

sudo security add-trusted-cert -d -r trustRoot \
-k /Library/Keychains/System.keychain ca/ca.crt
\end{verbatim}
