\section{Konfiguration einer PKI in OpenSSL}

\subsection{Einleitung}
In diesem Protokoll wird beschrieben, wie unter macOS eine vollständige Public Key Infrastructure (PKI) eingerichtet wird. Ziel ist es, ein eigenes Root-Zertifikat (CA) und ein signiertes Server-Zertifikat zu erstellen, dieses in einem lokalen Nginx-Webserver zu verwenden und die sichere Verbindung über HTTPS im Browser zu überprüfen. Abschließend wird gezeigt, wie das Server-Zertifikat über eine Certificate Revocation List (CRL) widerrufen wird.

\subsection{Voraussetzungen}
\begin{itemize}
    \item macOS (mit Terminal und OpenSSL)
    \item Homebrew zur Installation von Nginx
    \item Administratorrechte (für Keychain-Zugriff)
\end{itemize}

\subsection{Ordnerstruktur}
Im Benutzerverzeichnis wird folgender Projektordner angelegt:

\begin{tcolorbox}[colback=black!3!white, colframe=black!60!white]
    \begin{verbatim}
mkdir -p ~/pki-demo/{ca,server,crl}
cd ~/pki-demo
    \end{verbatim}
\end{tcolorbox}

\subsection{Erstellung der Root-CA}
Zuerst wird eine Root-CA erstellt, die als Vertrauensanker dient.

\subsubsection{Private Key}

\begin{tcolorbox}[colback=black!3!white, colframe=black!60!white]
    \begin{verbatim}
openssl genrsa -out ca/ca.key 4096
    \end{verbatim}
\end{tcolorbox}

\subsubsection{Self-signed Root-Zertifikat}

\begin{tcolorbox}[colback=black!3!white, colframe=black!60!white]
    \begin{verbatim}
openssl req -x509 -new -nodes -key ca/ca.key -sha256 -days 3650 \
  -out ca/ca.crt \
  -subj "/C=AT/ST=Niederoesterreich/L=Mariazell/O=C/OU=C/CN=Christian Wunder Root CA"
    \end{verbatim}
\end{tcolorbox}

Dieses Zertifikat wird später in die macOS-Keychain importiert, um von Chrome und Safari als vertrauenswürdig erkannt zu werden.

\subsection{Erstellung des Server-Zertifikats}

\subsubsection{Private Key}

\begin{tcolorbox}[colback=black!3!white, colframe=black!60!white]
    \begin{verbatim}
openssl genrsa -out server/server.key 2048
    \end{verbatim}
\end{tcolorbox}

\subsubsection{Konfigurationsdatei mit SAN}

\begin{tcolorbox}[colback=black!3!white, colframe=black!60!white]
    \begin{verbatim}
cat > server/openssl_server.cnf << 'EOF'
[ req ]
default_bits        = 2048
prompt              = no
default_md          = sha256
distinguished_name  = dn
req_extensions      = v3_req

[ dn ]
C  = AT
ST = Niederoesterreich
L  = Mariazell
O  = C
OU = C
CN = localhost

[ v3_req ]
subjectAltName = @alt_names

[ alt_names ]
DNS.1 = localhost
IP.1  = 127.0.0.1
EOF
    \end{verbatim}
\end{tcolorbox}

\subsubsection{CSR (Certificate Signing Request)}

\begin{tcolorbox}[colback=black!3!white, colframe=black!60!white]
    \begin{verbatim}
openssl req -new -key server/server.key -out server/server.csr \
  -config server/openssl_server.cnf
    \end{verbatim}
\end{tcolorbox}

\subsubsection{Signieren durch die Root-CA}

\begin{tcolorbox}[colback=black!3!white, colframe=black!60!white]
    \begin{verbatim}
openssl x509 -req -in server/server.csr -CA ca/ca.crt -CAkey ca/ca.key \
  -CAcreateserial -out server/server.crt -days 825 -sha256 \
  -extfile server/openssl_server.cnf -extensions v3_req
    \end{verbatim}
\end{tcolorbox}

\subsection{Import der Root-CA in macOS}
Damit Chrome und Safari dem Zertifikat vertrauen, muss die Root-CA als „System Root“ importiert werden:

\begin{tcolorbox}[colback=black!3!white, colframe=black!60!white]
    \begin{verbatim}
sudo security add-trusted-cert -d -r trustRoot \
  -k /Library/Keychains/System.keychain ca/ca.crt
    \end{verbatim}
\end{tcolorbox}
