\section{Firefox}

\subsection{Testumgebung}
Als Client-System wird ein Windows 11-Rechner verwendet, auf dem der Firefox-Browser zur Verbindung mit dem Webserver genutzt wird. Die Server-VM basiert auf Ubuntu mit einem installierten Nginx-Webserver, der die zuvor erstellten Zertifikate nutzt.

\begin{itemize}
    \item \textbf{Server:} Ubuntu (Nginx, IP: 192.168.3.210)
    \item \textbf{Client:} Windows 11 (Firefox)
    \item \textbf{Protokoll:} HTTPS (Port 443)
    \item \textbf{Zertifikate:}
    \begin{itemize}
        \item Root CA: \texttt{root\_CA.crt}
        \item Intermediate CA: \texttt{intermediate\_CA.crt}
        \item Server-Zertifikat: \texttt{server.crt}
    \end{itemize}
\end{itemize}

\subsection{Import der Zertifikate}
Damit Firefox dem Server-Zertifikat vertraut, muss die Root-Zertifizierungsstelle (Root CA) direkt im Browser importiert werden. Firefox verwendet im Gegensatz zu Chrome einen eigenen Zertifikatsspeicher, sodass der Import im Browser selbst erfolgen muss.

\subsubsection*{Vorgehen}
\begin{enumerate}
    \item Firefox öffnen und zu \textit{Einstellungen} navigieren.
    \item \textit{Datenschutz \& Sicherheit} auswählen und zum Abschnitt \textit{Zertifikate} scrollen.
    \item Auf \textit{Zertifikate anzeigen} klicken.
    \item \textit{Importieren} auswählen und die Dateien \texttt{root\_CA.crt} und \texttt{intermediate\_CA.crt} importieren.
    \item Beim Import sicherstellen, dass die Option
    \textit{„Dieser CA vertrauen, um Websites zu identifizieren“}
    aktiviert ist.
\end{enumerate}

\subsection{Verbindungstest}
Nach dem Import der Zertifikate muss der Browser neu gestartet werden. Anschließend kann die folgende URL aufgerufen werden:

\begin{verbatim}
https://192.168.3.210
\end{verbatim}

Wenn der Import erfolgreich war, zeigt Firefox die Verbindung als sicher an.
